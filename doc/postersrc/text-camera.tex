%!TEX root = /Users/paul/phd/talks/WD12poster/puokonui-poster.tex
\vspace{-1.75cm}
\begin{itemize}[itemsep=20pt]
	
\item[] Puoko-nui has two CCD detectors. The primary detector is a
1k\,$\times$\,1k pixel frame-transfer CCD that is part of a Princeton Instruments
Micromax camera system.  A smaller SBIG ST-402ME CCD is mounted in an
offset guiding position with a 2D slide mechanism, which allows a bright star
outside the main field of view to be imaged independently of the main detector
for telescope autoguiding. Both detectors connect to their respective control PCs
via USB.

\item[] The frame-transfer operation of the primary CCD effectively eliminates
readout deadtime, allowing the system to be run without a shutter. At the
maximum 1\,MHz readout rate, full-resolution exposures as short as 2 seconds
are possible. Faster rates can be achieved by binning pixels. The CCD is
thermoelectrically cooled to $-50^\circ$C to reduce thermal noise.

\item[] Our white dwarf observations use a broad blue band
BG40 filter to reduce the sensitivity to red sky photons.
The primary CCD is normally operated with 2\,$\times$\,2 pixel binning and 100\,kHz 
readout to minimize noise. 

\item[] Paired with the MJUO 1m telescope at f/8, the primary field of view is
5.7 square arcminutes. When operated with 2\,$\times$\,2 binning, the aggregate
pixels each image a 0.66 square arcsecond region of the sky.

\end{itemize}
\vspace{-1.75cm}